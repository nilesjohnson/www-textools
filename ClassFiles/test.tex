\documentclass[noocg]{wwwnotes2}  
%noocg by default. On Laptop ocgx is not installed?
%\usepackage[geom,slash,sets,norm,errorterm,pde,miscops,labeled,conv,roman]{www_math_commands}

\addbibresource{mixmaster.bib}

\usepackage{lipsum}

\title{Title}
\subtitle{Lecture Notes Given for XXXYYY}
\author{Willie Wai-Yeung Wong}

\begin{document}

\maketitle



\tableofcontents

\chapter{First chapter}

\lipsum[1-2]

\begin{align}
	A & = B' \desclabel{equation0}{Equation 0} \\
	C & = D \desclabel{equation1}{Equation 0}
\end{align}



\begin{equation}\desclabel{eq1}{$A = B$}
	A = B
\end{equation}

\begin{prop}\desclabel{someprop}{An unimportant proposition}
   Text
\end{prop}

\begin{thm}\desclabel{thm1}{This is a nonsense theorem with no content.}
	\lipsum[27]

	test
\end{thm}

\lipsum[1]

\begin{pfof}{thm1}
THis is how it goes.
\begin{thot}[Step 1]\desclabel{parastep}{In this step we check a foo is a bar}
	\lipsum[9-10]

	And this is sufficient.
\end{thot}
Then we can apply \desceqref{eq1} and \descthref{someprop} to get the final answer.
Here ends the proof.
\end{pfof}

\begin{rmk}\desclabel{rmk1}{Our proof is actually stronger}
	\lipsum[27]

	test
\end{rmk}

\begin{cor}\desclabel{cor1}{Here's an immediate corollary}
	\lipsum[24]

	text
\end{cor}

\begin{pfof}{cor1}
	By using \descthref{rmk1}, we see that \desceqref{eq1} in fact implies something else. Thus the arguments in \descthref{parastep} is in fact good enough to show the strengthened results. 
\end{pfof}

\begin{equation}
	C = D
\end{equation}

\section{A section on \TeX{}niques}

Before doing anything else, let us recall \descthref{thm1}. Also let's try \descthref{parastep}.

\lipsum[2]

\begin{equation}
	A = B
\end{equation}

\lipsum[3-8]



\section{Another section}

\lipsum[5-9]

Some text \cite{Wong2009}.


\chapter{Second chapter}


\lipsum[10-23]

\begin{equation}
K = \frac12 mv^2
\end{equation}

This is it.


\cleardoublepage
\printbibliography[heading=bibintoc]

\end{document}
