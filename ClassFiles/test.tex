\documentclass[noocg]{wwwnotes2}  
%noocg by default. On Laptop ocgx is not installed?
%\usepackage[geom,slash,sets,norm,errorterm,pde,miscops,labeled,conv,roman]{www_math_commands}

\addbibresource{mixmaster.bib}

\usepackage{lipsum}

\title{Title}
\subtitle{Lecture Notes Given for XXXYYY}
\author{Willie Wai-Yeung Wong}

\begin{document}

\maketitle



\tableofcontents

\chapter{First chapter}

\lipsum[1-2]

\csname\InTheoType Keyword\endcsname

\begin{equation}\label{eq1}
	A = B
\end{equation}

\begin{prop}\label{someprop}
   Text. \csname\InTheoType Keyword\endcsname
\end{prop}



\begin{thm}\desclabel{thm1}{This is a nonsense theorem. Let's make the description longer to see what we can see. Let us make it longer to see if we can see other interesting things about it.}
	\lipsum[27]

	test \csname\InTheoType Keyword\endcsname
\end{thm}

\lipsum[1]

\begin{pfof}{thm1}
THis is how it goes.
\begin{thot}[Step 1]\desclabel{parastep}{Here is another nonsense stuff. This is supposed to be the first step to proving. But who knows what this is actually saying.}
	\lipsum[9-10]

	And this is sufficient.
\end{thot}
Then we can apply \desceqref{eq1} and \descthref{someprop} to get the final answer.
Here ends the proof.
\end{pfof}

Let's refer to \thref{parastep}

\begin{rmk}
	\lipsum[27]

	test
\end{rmk}

\begin{equation}
	C = D
\end{equation}

\section{A section on \TeX{}niques}

Before doing anything else, let us recall \descthref{thm1}. Also let's try \descthref{parastep}.

\lipsum[2]

\begin{equation}
	A = B
\end{equation}

\lipsum[3-8]



\section{Another section}

\lipsum[5-9]

Some text \cite{Wong2009}.


\chapter{Second chapter}

Test \theequation to see what number would show up.

\lipsum[10-23]

\begin{equation}
K = \frac12 mv^2
\end{equation}

This is it.


\cleardoublepage
\printbibliography[heading=bibintoc]

\end{document}
