\documentclass[noocg]{wwwnotes2}  
%noocg by default. On Laptop ocgx is not installed?
%\usepackage[geom,slash,sets,norm,errorterm,pde,miscops,labeled,conv,roman]{www_math_commands}

\usepackage{lipsum}

\title{Title}
\subtitle{Lecture Notes Given for XXXYYY}
\author{Willie Wai-Yeung Wong}

\begin{document}

\maketitle

\tableofcontents

\chapter{First chapter}

\lipsum[1-2]

Test \marginnote{Something}

\begin{equation}
	A = B
\end{equation}

\begin{thm}\label{thm1}
	\lipsum[27]

	test
\end{thm}

\lipsum[1-3]

\begin{pfof}{thm1}
THis is how it goes.
\begin{paras}[Step 1]\label{parastep}
	\lipsum[9-10]

	And this is sufficient.
\end{paras}
Here ends the proof.
\end{pfof}

Let's refer to \thref{parastep}

\begin{paras}
	\lipsum[2-3]

	Something
\end{paras}

\begin{rmk}
	\lipsum[27]

	test
\end{rmk}

\begin{equation}
	C = D
\end{equation}

\section{A section}
\lipsum[2]

\begin{equation}
	A = B
\end{equation}

\lipsum[3-8]



\section{Another section}

\lipsum[5-9]

\chapter{Second chapter}

Test \theequation to see what number would show up.

\lipsum[10-23]

\begin{equation}
K = \frac12 mv^2
\end{equation}

This is it.



\cleardoublepage
\phantomsection
\addcontentsline{toc}{part}{Bibliography}
\bibliographystyle{amsalpha}
\bibliography{mixmaster}
\end{document}
