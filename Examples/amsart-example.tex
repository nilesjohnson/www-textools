\documentclass[amssymb,amsfonts]{wwwnotes2}
\usepackage{amssymb, latexsym}
\usepackage[T1]{fontenc}
% So that the accented characters are dealt with correctly
\usepackage[intlimits,sumlimits,noDcommand,narrowiints]{kpfonts}
\usepackage[english]{babel}
\usepackage{microtype}
\usepackage{svn}
\usepackage[notref,notcite]{showkeys}
%\usepackage{changes}
% The changes package allows editing marks (addition, deletion,
% replacement) with attribution to author(s). See CTAN
%\usepackage{empheq}
% Empheq 'emphasizes equations', and gives a visual markup extension
% to amsmath. In particular you can have a left braced set of
% equations number individually. See CTAN for details
\usepackage{tensor}
% The tensor package allows left indices and auto-justified/spaced
% sub/super scripts. See Documentation in CTAN
%\usepackage{bbm}
% Provides blackboard characters for other than upper-case latin
\usepackage{constants}
% Constants provide a way to automatically number constant/error
% terms.
%\usepackage[pdfusetitle]{hyperref}
% Remember to replace when done!
%\usepackage{www_math_commands}

\usepackage{lipsum}



%%% In the code below author is not the same as the person making the
%%% committ; see http://git-scm.com/docs/git-commit option --author
%%% for more information
\begingroup
\catcode`\%=12
\catcode`\$=12
\gdef\gitAuthorEmail{\url{$Format:%aE$}}
\gdef\GitPageFooter{$Format: v:\texttt{%h %d}; last edit: %aN on %ai.$}
\endgroup


%\renewcommand{label}[1]{\desclabel{#1}{#1}}

\begin{document}
\title[short title]{long title}
\author[WWY Wong]{Willie Wai Yeung Wong}
%\address{Michigan State University}
\thanks{other thanks}
\thanks{\GitPageFooter}
%\email{wongwwy@member.ams.org}
%\subjclass[2000]{MSC, look up on AMS}

\begin{abstract}
\end{abstract}




\maketitle



\tableofcontents

\chapter{First chapter}

\lipsum[1-2]

\begin{align}
C &= D \desclabel{eq000}{$C = D$}\\
C' &= D' \desclabel{eq001}{$C' = D'$}
\end{align}

\thref{eq001}

\begin{subequations}
\begin{align}
E &= F \desclabel{eq002}{Another equation}\\
E' &= F' \desclabel{eq003}{Another equation 2}
\end{align}

some text

\begin{equation}\desclabel{eq1}{$A = B$}
	A = B
\end{equation}
\end{subequations}

\begin{prop}\desclabel{someprop}{An unimportant proposition}
   Text
\end{prop}

\begin{thm}\desclabel{thm1}{This is a nonsense theorem with no content.}
	\lipsum[27]

	test
\end{thm}

\lipsum[1]

\begin{pfof}{thm1}
THis is how it goes.
\begin{thot}[Step 1]\desclabel{parastep}{In this step we check a foo is a bar}
	\lipsum[9-10]

	And this is sufficient.
\end{thot}
Then we can apply \desceqref{eq1} and \descthref{someprop} to get the final answer.
Here ends the proof.
\end{pfof}

\begin{rmk}\desclabel{rmk1}{Our proof is actually stronger}
	\lipsum[27]

	test
\end{rmk}

\begin{cor}\desclabel{cor1}{Here's an immediate corollary}
	\lipsum[24]

	text
\end{cor}

\begin{pfof}{cor1}
	By using \descthref{rmk1}, we see that \desceqref{eq1} in fact implies something else. Thus the arguments in \descthref{parastep} is in fact good enough to show the strengthened results. 
\end{pfof}

\begin{equation}
	C = D
\end{equation}

\section{A section on \TeX{}niques}

Before doing anything else, let us recall \descthref{thm1}. Also let's try \descthref{parastep}.

\lipsum[2]

\begin{equation}
	A = B
\end{equation}

\lipsum[3-8]



\section{Another section}

\lipsum[5-9]

Some text \cite{Wong2009}.


\chapter{Second chapter}


\lipsum[10-23]

\begin{equation}
K = \frac12 mv^2
\end{equation}

This is it.


\bibliographystyle{amsalpha}
\bibliography{mixmaster.bib}

\end{document}
